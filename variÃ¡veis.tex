\documentclass[12pt]{article} 
\usepackage{amsmath,amsfonts} 
\usepackage{graphicx}
\graphicspath{ {./images/} }
\usepackage[brazilian]{babel}
\usepackage{natbib}

%\usepackage{biblatex}
\bibliographystyle{unsrtnat}
\usepackage{aeguill}
\usepackage{amsmath}



\begin{document} 

%%%%%%%%%%%%%%%%%%%
A: 
B: 
R: 

\\
\begin{equation} \label{A1_B5} %1 pus no word    %1
X_{t+1}  =  \mu + \Phi X_{t} + v_{t+1} \qquad \qquad  u_{t} \sim \mathcal{N}(0,\,\Sigma)\,
\end{equation}
\\

\\
X_{t} se refere a um vetor $Kx1$ de proxies observáveis ou fatores de risco latentes.
A: =
B: x    !
R: pc5_ini
\\

\\
\mu
A: =
B: c
R: 
\\

\\
$Phi$
A: =
B: rho  !
R: 
\\

\\
$v_t$ inovação no estado
A: =
B: Sigma*u
R: v_hat
\\


\Sigma é a matriz de covariância desse termo $v_t$
A: =
B: 1
R: Sigma_hat
\\

\\
\begin{equation} \label{A21_B6}  pus no word        %2
\ln P_t(n)  =  A_n + B^{\prime}_n X_t + u_t(n) \,
\end{equation}
\\

\\
lnP_t(n) 

A: =
B: f_t^h    !
R: -(mns.lnP)
\\


\\
A_n

A: =
B: \alpha_h
R: 
\\

\\
B^{\prime}_n


A: =
B: \beta
R:
\\

\\
u_t(n)

A: =
B: não tem
R: 
\\


\\

Substituindo \ref{A21_B6} em \ref{A1_B5} e recursivamente, temos:    %3

\begin{equation} % R(1) 
\ln P_{t+1}(h-1)  =  A(n-1) + B^{\prime}_{n-1} (\mu + \Phi X_t + v_{t+1}) + u(h-1)_{t+1} \,
\end{equation}
\\

\\
Assim, é possível derivar o excesso de retorno de manter um título por $h$ períodos, $r*^{h-1}_{t+1}$ 

\begin{equation} % A(6)      %4
r^{*}_{t+1}(n-1)  =  \ln P_{t+1}(h-1) - \ln P_{t+1}(h-1) - rf_t \,
\end{equation}

Onde $rf_t$ refere-se à taxa livre de risco 
\\



\\
r^{*}_{t+1}(n-1)

A: =
B: f_{t+1}^{h-1} - f_t^h
R: r_xs 
\\

\\
rf_t

A: r_t
B: $f^1_t$ 
R: rf
\\

\\
\\
\begin{equation} \label{R(2)}          %5
rx_{t+1}(n-1)  = \alpha_{h-1} + \beta^{\prime}_{h-1} (c+\rho x_t + u_{t+1}) + v^{h-1}_{t+1} -\alpha_h - \beta^{\prime}{_h} x_t - v^h_t - r^*_t\,
\end{equation}

\begin{equation} \label{A25_B7}         %6
A_n=A_n-1 + B^{\prime}_{n-1}(\mu-\lambda_0) + \frac{1}{2}(B^{\prime}_{n-1} \Sigma B^{\prime}_{n-1} + \sigma^2) - \delta_0
\end{equation}
\\
\\

\\

\\

\\
\alpha


A: !
B: 
R: 
\\

\\
\beta


A: !
B: 
R:
\\

\\
v

A: !
B: 
R: 
\\v


\\
---------------------------------------------

\\


%%%%%%%%%%%%%%%%%


$F^h_t$ representa o contrato futuro negociado no tempo 

\\$t$ com maturidade 

\\$h$
\\
\\

$S_{t}$ o preço spot do ativo no período $t$
\\
\\

$M_{t+h}$, chamada “fator de desconto estocástico”
\\
\\

 $RP^h_t$ prêmio de risco
\\
\\

$\gamma$ captura alguns erros sistemáticos de expectativas 
\\
\\

O DFM expressa um vetor 
\\

$X_t$ de $N$ variáveis de séries temporais observadas (N=17 séries, Cada X é a série de uma  maturidade)  
\\

$X_{k,t}$ representa um vetor que contém $K$ variáveis preditivas, disponíveis aos agentes no tempo $t$: 
\\

$ X_{k,t}$ como fatores de risco 
\\    



$q$ de fatores não-observáveis ou latentes $f_t$ 
  \\
 
 $e_t$componente idiossincrático de média zero
\\

  $\lambda (L)$ matrizes $N \times q$ polinomiais de defasagem\ 
  \\
  
 $\Psi (L)  $ as matrizes $q \times q$ polinomiais de defasageme  
 \\
 
 $\eta$ é o vetor $q \times 1$ de inovações de média zero (serialmente não correlacionadas) aos fatores. 
 \\
 
 $\eta_t$ são distúrbios Gaussianas
 \\
 
$\eta^{\prime}_{h-1} = \beta^{\prime}_{h-1} \lambda_1 + \delta_1$
\\

 As perturbações idiossincráticas são assumidas como não correlacionadas com as inovações dos fatores em todos os avanços e defasagens, isto é $E(e_t) \eta^{\prime}_{t-k} = 0$ para todo $k$. 
\\

$r$ fatores estáticos $F_t$ 
\\

 $p$ o grau da matriz polinomial de defasagem
\\

 $F_t = (f^{\prime}_{t}$ , $f^{\prime}_{t-1} ,...,f^{\prime}_{t-p})$ um vetor $r \times 1$ dos fatores estáticos. 
\\

 $\Lambda = (\lambda_0 , \lambda_1 ,...,  \lambda_p )$,  onde 
 \\
 
 $\lambda_h$ é a matrix $N \times q$ de coeficientes na $h$-ésima defasagem em $\lambda (L)$. 
 \\


 $\Phi (L)$ uma matriz de 1s, 0s e os elementos de $\Psi (L)$
\\

 $G = [I_q \quad 0_{q \times (r - q) } ]^{\prime}$.
\\

$v_t$ são distúrbios Gaussianas
\\

 $\alpha^F_i = \Lambda_i \Phi (L) - \delta_i (L) \Lambda_i$. 
\\

$Q$ é qualquer matriz $r \times r$ inversível
\\

 $\overline{A} = N^{-1} \sum_{i=1}^{N} A_i$, a média cross-sectional
\\

$V_t (\Lambda , F) = \frac{1}{N T} \sum_{t=1}^{T} (X_t - \Lambda F_t)^{\prime} (X_t - \Lambda F_t)$
\\

$m $ fatores latentes ou observados $x_t$ determina conjuntamente todos os preços de ativos na economia de uma forma internamente consistente
\\

\label{A1_B5}   c+\rho 
\\

} \label{A21_B6} \alpha_h + \beta^{\prime}{_h}  + v^h_t \,
\\

com factor loadings dados por:
\\

$A_n=A_n-1 + B^{\prime}_{n-1}(\mu-\lambda_0) + \frac{1}{2}(B^{\prime}_{n-1} \Sigma B^{\prime}_{n-1} + \sigma^2) - \delta_0$
\\

\label{A26_B8} \Phi
\\

sob uma lei de movimento modificada para os fatores:
\\

$x_{t+1}  =  c^Q + \rho^Q x_{t}+u^Q_{t+1} \qquad \qquad  u^Q_{t} \sim \mathcal{N}(0,\,\Sigma_{m})$
\\


\\
$c^Q = c - \lambda$ e $\rho^Q = \rho- \Lambda$ ajustes que resultam da aversão ao risco. 


$k_{h-1} = \beta ^{\prime}_{h-1}\lambda_0 - \frac{1}{2}(\beta^{\prime}_{h-1} \Sigma \beta_{h-1} + \sigma^2) - \delta_o$
\\


$\epsilon^{h-1}_{t+1} = \beta^{\prime}_{h-1} u_ {t+1} + v^ {h-1}_{t+1} -v^h_t$
\\

$\delta_0$ e 
\\

$\delta_1$ são obtidos por MQO: $r^F_t  = \delta_0 + \delta^{\prime}_1 x_t$
\\

$r^{"*"}_t$ retorno em excesso
\\

\citep{bmstr2021}
\end{document}
