\documentclass[12pt]{article} 
\usepackage{amsmath,amsfonts} 
\usepackage{graphicx}
\graphicspath{ {./images/} }
\usepackage[brazilian]{babel}
%\usepackage{natbib}
\bibliographystyle{chicago}
\usepackage{aeguill}
\usepackage{amsmath}

\begin{document} 

\section{Derivação}
\label{sub_sect}


%\begin{equation} \label{B1}
%F^h_t = \frac{(S_{t+h} - S_t)}{S_t} = \alpha + \beta \frac {(F^h_t - S_t)}{S_t} + \epsilon_{t+h}
%\end{equation}

O uso de contratos futuros como preditores do preço spot de ativos baseia-se na hipótese de expectativas racionais em mercados financeiros eficientes, os quais inviabilizam ganhos com arbitragem. Nestas circunstâncias, presume-se que os agentes negociam de modo que se gerem preços de equilíbrio nos contratos futuros que exprimam suas reais expectativas quanto ao preço spot no futuro, dado o conjunto de informações disponíveis no presente. Em outras palavras, $ F^h_t = E_t [S_{t+h}]$, onde $F^h_t$ representa o contrato futuro negociado no tempo $t$ com maturidade $h$, $S_{t+h}$ o preço spot do ativo no período $t+h$, e $E_t$ operador de esperança, representando as expectativas dos agentes no tempo $t$   . 



\begin{equation} \label{A1_B5} %1 pus no word
X_{t+1}  =  \mu + \Phi X_{t} + v_{t+1} \qquad \qquad  u_{t} \sim \mathcal{N}(0,\,\Sigma)\,
\end{equation}

\begin{equation} \label{A21_B6} %2 pus no word
\ln P_t(n)  =  A_h + B^{\prime}_n X_t + u_t(n) \,
\end{equation}

Substituindo \ref{A21_B6} em \ref{A1_B5} e recursivamente, temos: %3

\begin{equation} % R(1) 
\ln P_{t+1}(h-1)  =  A(n-1) + B^{\prime}_{n-1} (\mu + \Phi X_t + v_{t+1}) + u_{t+1}(h-1) \,
\end{equation}

Assim, é possível derivar o excesso de retorno de manter um título por $h$ períodos, $rx_{t+1}(n-1)$ 

\begin{equation} % A(6) #4
rx_{t+1}(n-1)  =  \ln P_{t+1}(h-1) - \ln P_{t+1}(h-1) - r^*_t \,
\end{equation}

Onde $r^*_t$ refere-se à taxa livre de risco %representada como em Baumeister (2022) $f^1_t$ 

\begin{equation} \label{R(2)}
rx_{t+1}(n-1)  = \alpha_{h-1} + \beta^{\prime}_{h-1} (c+\rho x_t + u_{t+1}) + v^{h-1}_{t+1} -\alpha_h - \beta^{\prime}{_h} x_t - v^h_t - r^F_t\,
\end{equation}

\begin{equation} \label{A25_B7}
A_n=A_n-1 + B^{\prime}_{n-1}(\mu-\lambda_0) + \frac{1}{2}(B^{\prime}_{n-1} \Sigma B^{\prime}_{n-1} + \sigma^2) - \delta_0
\end{equation}


onde $A_h$ depende de $A_{h-1}$ para restringir arbitragem. 
\\

\begin{equation} \label{A26_B8}
B^{\prime}_n = B^{\prime}_{n-1}(\Phi-\lambda_1) - \delta^{\prime}_1
\end{equation}
\\

Aplicando \ref{A25_B7} e \ref{A26_B8} a \ref{R(2)}, temos:


\begin{equation}  \label{R3}
\begin{split}
rx_{t+1}(n-1)  = \beta^{\prime}_{h-1} u_{t+1} + \beta^{\prime}_{h-1}\lambda_0 - \frac{1}{2}(\beta^{\prime}_{h-1} \Sigma \beta_{h-1} + \sigma^2) - \delta_o \\ + \beta^{\prime}_{h-1} \lambda_1 x_t + \delta_1 x_t + v^{h-1}_{t+1} - v^h_t
\end{split}
\end{equation}
\\

Considerando:

\begin{equation}
k_{h-1} = \beta^{\prime}_{h-1}\lambda_0 - \frac{1}{2}(\beta^{\prime}_{h-1} \Sigma \beta_{h-1} + \sigma^2) - \delta_o
\end{equation} 

\begin{equation} 
\eta^{\prime}_{h-1} = \beta^{\prime}_{h-1} \lambda_1 + \delta_1
\end{equation} 

\begin{equation} 
\epsilon^{h-1}_{t+1} = \beta^{\prime}_{h-1} u_{t+1} + v^{h-1}_{t+1} -v^h_t
\end{equation} 


Temos:

\begin{equation} \label{B9}
rx_{t+1}(n-1) = k_{h-1} + \eta^{\prime}_{h-1} x_t + \epsilon^{h-1}_{t+1} 
\end{equation} 
\\

onde $\delta_0$ e $\delta_1$ são obtidos por MQO:


\begin{equation} 
r^F_t = \delta_0 + \delta^{\prime}_1 x_t
\end{equation} 

\end{document} 
